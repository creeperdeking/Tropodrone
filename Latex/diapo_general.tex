\documentclass{beamer}

\usepackage{gensymb}
\input{entete_beamer_pdflatex}
\usepackage{listings}
\usepackage{transparent}
\usepackage[babel=true]{csquotes}
\lstset{language=Python, tabsize=2, breaklines=true, showstringspaces=false}

\definecolor{RedT}{rgb}{1, .7, .7}
\definecolor{GreenT}{rgb}{.7, 1, .7}
\definecolor{OrangeT}{rgb}{1, 1, .7}

\useoutertheme{infolines}
\setbeamersize{text margin left=1cm,text margin right=1cm}

\title{Rapport de projet de SI}
\subtitle{Tropodrone}
\author{Gueydan Noé, Manceau Thibaut, Gros Alexis, Porteries Tristan}

\usebackgroundtemplate%
{%
    {\transparent{0.2}\includegraphics[width=\paperwidth,height=\paperheight]{../Images/structure1_1.PNG}}%
}

\begin{document}

\begin{frame}
  \titlepage
\end{frame}

\begin{frame}
    \frametitle{Sommaire}
    \begin{multicols}{2}
      {
		\setcounter{tocdepth}{1}
        \tableofcontents
      }
    \end{multicols}
\end{frame}

\section{Courte présentation}

\subsection{But du projet}
\begin{frame}{But du projet}
 Créer une structure composée de ballons contenant un gaz plus léger que l’air qui contre une partie du poids d'un drone.
 Augmenter l’autonomie en réduisant la vitesse des moteurs.
\end{frame}

\subsection{Contraintes imposées au projet}
\begin{frame}{Contraintes imposées au projet}
  \begin{itemize}
    \item être simple d’utilisation, garder la manœuvrabilité du drone au possible, voler le plus longtemps possible~;
    \item consommer le moins d’énergie possible, ne pas présenter de danger pour le public et économiser le gaz et les matériaux de fabrication~;
    \item le modèle du drone imposé~;
    \item rayon d’action de minimum 15 mètres.
  \end{itemize}
\end{frame}


