\documentclass{beamer}

\usepackage{gensymb}
\input{entete_beamer_pdflatex}
\usepackage{listings}
\usepackage{transparent}
\usepackage[babel=true]{csquotes}
\lstset{language=Python, tabsize=2, breaklines=true, showstringspaces=false}

\definecolor{RedT}{rgb}{1, .7, .7}
\definecolor{GreenT}{rgb}{.7, 1, .7}
\definecolor{OrangeT}{rgb}{1, 1, .7}

\useoutertheme{infolines}
\setbeamersize{text margin left=1cm,text margin right=1cm}

\title{Rapport de projet de SI}
\subtitle{Tropodrone}
\author{Gueydan Noé, Manceau Thibaut, Gros Alexis, Porteries Tristan}

\usebackgroundtemplate%
{%
    {\transparent{0.2}\includegraphics[width=\paperwidth,height=\paperheight]{../Images/structure1_1.PNG}}%
}

\begin{document}

\begin{frame}
  \titlepage
\end{frame}

\begin{frame}
    \frametitle{Sommaire}
    \begin{multicols}{2}
      {
		\setcounter{tocdepth}{1}
        \tableofcontents
      }
    \end{multicols}
\end{frame}

\section{Courte présentation}

\subsection{But du projet}
\begin{frame}{But du projet}
 Créer une structure composée de ballons contenant un gaz plus léger que l’air qui contre une partie du poids d'un drone.
 Augmenter l’autonomie en réduisant la vitesse des moteurs.
\end{frame}

\subsection{Contraintes imposées au projet}
\begin{frame}{Contraintes imposées au projet}
  \begin{itemize}
    \item être simple d’utilisation, garder la manœuvrabilité du drone au possible, voler le plus longtemps possible~;
    \item consommer le moins d’énergie possible, ne pas présenter de danger pour le public et économiser le gaz et les matériaux de fabrication~;
    \item le modèle du drone imposé~;
    \item rayon d’action de minimum 15 mètres.
  \end{itemize}
\end{frame}




\section{Aérodynamisme}

\begin{frame}{Objectif}
 But : que le drone soit le moins sensible à l'air possible. \\
 \textbf{Garder la manœuvrabilité du drone au possible}
\end{frame}

\subsection{Équation de traînée}
\begin{frame}{Équation de traînée}
  \begin{center}
		\includegraphics[width=5cm]{../Images/portance.jpg}
	\end{center}
 Force de traînée matérialisée par l'équation \\
 \begin{center}
  \boxed{\displaystyle{\frac12 \times \rho \times S \times Cx \times V^2}}
 \end{center}
 Avec~:
 \begin{itemize}
  \item $\rho$ la masse volumique du fluide dans lequel a lieu le déplacement en $kg.m^{-3}$~;
  \item S la surface~;
  \item Cx le coefficient de traînée~;
  \item V la vitesse relative du mobile par rapport au fluide en $m.s^{-1}$.
 \end{itemize}
\end{frame}

\subsection{Problèmes et solutions techniques}
\begin{frame}{Problème des ballons}
  \begin{itemize}
	\item 3 ballons de 0.25 m³ : surface à réduire
  \item Cx différent selon position des ballons face au flux d'air
\end{itemize}
\end{frame}

\begin{frame}{La forme en triangle}
	Permet de créer un couloir dans lequel le vent s'engouffre.
  \begin{center}
		\includegraphics[width=10cm]{../Images/Capture.PNG}
	\end{center}
\end{frame}

\begin{frame}{Gouvernail}
	Le gouvernail permet d'orienter les ballons en fonction de la direction du drone.\\
  Assimilable aux conformations d'une molécule~:
  \begin{center}
		\includegraphics[width=10cm]{../Images/conformations.png}
	\end{center}
\end{frame}

\subsection{Résolution de l'équation}
\begin{frame}{Résolution pour une sphère}
\begin{center}
 \boxed{\displaystyle{Re = \frac{V \times L}{\nu}}}
\end{center}
Avec~:
\begin{itemize}
 \item $Re$ la masse volumique du fluide dans lequel a lieu le déplacement en $kg.m^{-3}$~;
 \item V la vitesse relative du mobile par rapport au fluide en $m.s^{-1}$;
 \item L la dimension caractéristique (ici le diamètre de la sphère)~;
 \item $\nu$ la viscosité cinématique du fluide~;
\end{itemize}
\end{frame}

\begin{frame}{Résolution pour une sphère - Algorithme}
  \begin{center}
    \lstset{language={Python}}
    \lstinputlisting{../Aerodynamisme/Programmes/calculTraineeReduit.py}
  \end{center}
\end{frame}

\begin{frame}{Résolution pour une sphère - Algorithme}
  \begin{center}
    \lstset{language={Python}}
    \lstinputlisting{../Aerodynamisme/Programmes/calculTraineeReduit2.py}
  \end{center}
\end{frame}

\begin{frame}{Résolution sous SW - 1}
  \begin{center}
    \includegraphics[width=10cm]{../Images/objectifsSW.png}
    \captionof{figure}{Objectif de calcul pour déterminer F.}
  \end{center}
\end{frame}

\begin{frame}{Résolution sous SW - 2}
  \begin{center}
    \includegraphics[width=5cm]{../Images/expressionCX.png}
    \captionof{figure}{Objectif de calcul pour déterminer Cx.}
  \end{center}
\end{frame}

\begin{frame}{Résolution sous SW - 3}
	\begin{center}
		\begin{tabular}{c}
      \includegraphics[width=10cm]{../Images/Capture.PNG} \\
			\includegraphics[width=8cm]{../Images/resultatsSimulationSW.png}
		\end{tabular}
	\end{center}
\end{frame}

\begin{frame}{Conclusion - Données de SW}
	\begin{center}
    \includegraphics[width=10cm]{../Images/resultatsSW.png} \\
	\end{center}
\end{frame}

\begin{frame}{Conclusion}
  Le gouvernail présente un réel intérêt.\\
  Le drone peut lutter jusqu'à une vitesse de 20 km.h-¹
\end{frame}

\begin{frame}{Conclusion}
  \begin{center}
  FIN
  \end{center}
\end{frame}

\end{document}
