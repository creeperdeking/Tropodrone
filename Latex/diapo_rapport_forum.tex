\documentclass{beamer}

\usepackage{gensymb}
\input{entete_beamer_pdflatex}
\usepackage{listings}
\usepackage[babel=true]{csquotes}
\lstset{language=Python, tabsize=2, breaklines=true, showstringspaces=false}

\useoutertheme{infolines}
\setbeamersize{text margin left=1cm,text margin right=1cm}

\title{Rapport de projet de SI}
\subtitle{Tropodrone}
\author{Gueydan Noé, Manceau Thibaut, Gros Alexis, Porteries Tristan}

\begin{document}

\begin{frame}
  \titlepage
\end{frame}

\begin{frame}
    \frametitle{Sommaire}
    \begin{multicols}{2}
      {
		\setcounter{tocdepth}{1}
        \tableofcontents
      }
    \end{multicols}
\end{frame}

\section{Introduction}

\subsection{Idée}

\begin{frame}{}
 Fusionner un drone avec un dirigeable.
 
 Pourquoi~?~:
 \begin{itemize}
  \item augmenter l'autonomie du drone~;
  \item reduir la dangerosité du drone~;
  \item augmenter la charge utile.
 \end{itemize}

\end{frame}

\subsection{Contraintes et besoins}

\begin{frame}{}
	\begin{itemize}
		\item garder la manoeuvrabilité du drone~;
		\item porter une charge utile~;
		\item avoir des ballons rentables~;
		\item Augmenter l’autonomie, la sécurité d’un drone de petite taille.
	\end{itemize}

\end{frame}


\subsection{Objectifs}

\begin{frame}{Objectifs du projet}
  Créer une structure composée de ballons contenant un gaz plus léger que l’air qui soutient une partie ou la totalité du poids d'un drone.
\end{frame}


\section{Élévation}

\subsection{Gaz}

\begin{frame}{}
 
\end{frame}


\subsection{Dilatation}

\begin{frame}{}
 
\end{frame}


\section{Ballon}

\subsection{Matériaux}

\begin{frame}{}
 
\end{frame}


\subsection{Colles}

\begin{frame}{}
 
\end{frame}


\subsection{Forme}

\begin{frame}{}
 
\end{frame}


\subsection{Protocole}

\begin{frame}{}
 
\end{frame}


\section{Structure}

\subsection{Caractéristiques}

\begin{frame}{}
 
\end{frame}


\subsection{Ancienne structure}

\begin{frame}{}
 
\end{frame}


\subsection{Nouvelle structure}

\begin{frame}{}
 
\end{frame}


\section{Aérodynamisme}

\subsection{Simulation}

\begin{frame}{}
 
\end{frame}


\end{document}
